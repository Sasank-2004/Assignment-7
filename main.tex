\documentclass[journal,twocolumn]{IEEEtran}

\usepackage[utf8]{inputenc}
\usepackage{graphicx}
\usepackage{amssymb}
\usepackage{mathtools}
\usepackage{amsmath}
\providecommand{\pr}[1]{\ensuremath{\Pr\left(#1\right)}}
\providecommand{\cbrak}[1]{\ensuremath{\left\{#1\right\}}}

\title{Assignment 7}
\author{Gollapudi Sasank CS21BTECH11019}

\begin{document}
\maketitle
\section*{Question : }
Suppose box 1 contains $a$ white balls and $b$ black balls , and box 2 contains $c$ white balls and $d$ black balls.  One ball of unknown color is transferred from the first box into the second one and then a ball is drawn from the latter. What is the probability that it will be a white ball? 
\section*{Solution : }
If no ball is transferred from the first box into the second box, the probability of obtaining a white ball from the second one is simply $\frac{c}{c + d}$. In the present case, a ball is first transferred from box 1 to box 2 and there are only two mutually exclusive possibilities  for this event-the transferred ball is either a white ball or a black ball. Let the random variable $X$ denote the following : \\
$X = 0$ : transferred ball is white. \\
$X = 1$ : transferred ball is black. \\
The events $X=0$ and $X=1$ together form a partition $S$ , $S = (X=0) + (X=1)$ \\
From the given data 
\begin{align}
    \pr{X=0} &= \frac{a}{a+b} \\
    \pr{X=1} &= \frac{b}{a+b} 
\end{align}
After one ball is transferred at random from box 1 to box 2 Let the random variable $Y$ denote the following : \\
$Y = 0$ : The ball drawn at random from box 2 is white. \\
$Y = 1$ : the ball drawn at random from box 2 is black. \\
\begin{align}
    \pr{Y=0|X=0} &= \frac{c+1}{c+d+1} \\
    \pr{Y=0|X=1} &= \frac{c}{c+d+1} \\
    \pr{Y=1|X=0} &= \frac{d}{c+d+1} \\
    \pr{Y=1|X=1} &= \frac{d+1}{c+d+1} 
\end{align}
We have to find the value of $\pr{Y=0}$ \\
\begin{align}
    \pr{Y=0} &= \pr{(Y=0)(S)} \\
   \Rightarrow \pr{Y=0} &= \pr{(Y=0)((X=0)+(X=1))} \\
    \Rightarrow \pr{Y=0} &= \pr{((Y=0)(X=0)) + ((Y=0)(X=1))} 
\end{align}
Clearly the two events $(Y=0)(X=0)$ and $(Y=0)(X=1)$ are mutually exclusive because when we transfer one ball from box 1 to box 2 it must be either WHITE or BLACK.\\
\begin{align}
    \Rightarrow \pr{Y=0} &= \pr{(Y=0)(X=0)} + \pr{(Y=0)(X=1)}
    \end{align}
    \begin{multline}
    \Rightarrow \pr{Y=0} = \pr{X=0} \pr{Y=0|X=0} \\ +\pr{X=1}\pr{Y=0|X=1} 
    \end{multline}
    \begin{align}
    \Rightarrow \pr{Y=0} &= \frac{a(c+1)}{(a+b)(c+d+1)} + \frac{bc}{(a+b)(c+d+1)} \\
    \Rightarrow \pr{Y=0} &= \frac{ac+a}{(a+b)(c+d+1)} + \frac{bc}{(a+b)(c+d+1)} \\
    \Rightarrow \pr{Y=0} &= \frac{ac+bc+a}{(a+b)(c+d+1)}
\end{align}
\end{document}